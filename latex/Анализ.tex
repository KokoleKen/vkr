\section{Анализ предметной области}
\subsection{История создания трёхмерной компьютерной графики}

Впервые трёхмерная компьютерная графика была реализована в 1960-х годах, когда Айван Сазерленд и Дэвид Эванс основали первую в мире кафедру компьютерной графики в университете Юты, США. Основой трёхмерной компьютерной графики стала Евклидова геометрия, а также все математические открытия, которые были сделаны до XX века. Первые трёхмерные изображения состояли из множества точек и кривых, определяемых математическим уравнением. Именно в то время появился прародитель для всех современных 3D-редакторов - программа Sketchpad.

Развитие трёхмерной графики шло быстрым ходом, и уже совсем скоро вместо векторной графики появилась возможность создавать и трёхмерные растровые изображения, а также использовать различные текстуры для объектов, обрабатывать их тени, и наконец - проводить компиляцию всего кода в готовое изображение.
\subsection{Основные принципы работы трёхмерной графики}

Главным основоположником систематизации математических аксиом и теорем в области геометрии был Евклид. Именно его труды способствовали разработке и технологическому прорыву трёхмерной графики в XX веке. Но за её развитием стоит не только сам Евклид, а также и другие великие умы и открытия, как например, формулы Виета для нахождения корней квадратичного уравнения, благодаря чему в символьном анализе алгебры неизвестные переменные обозначаются как x, y и z, а коэффициенты - a, b и c. А основой отсчёта  пространства стала система трёхмерных координат Декарта.

Также незаменимый влкад в разработку трёхмерной графики и геометрии внесли Российские учёные XX века - Борис Делоне и Георгий Вороной. Борис Делоне предложил метод «Триангуляции Делоне», которая стала основой формирования граней трёхмерных моделей, а Георгий Вороной создал «Диаграмму Вороного», которая используется до сих пор в картографических софтах, дизайне и трёхмерной графике.

Любая трёхмерная модель объекта математически находится в трёх измерениях. И если, чтобы отрисовать изображение объекта в двух измерениях компьютерной мощи требовалось не так много, то с добавлением третьего измерения - оси Z, требовательность к производительности компьютерного железа резко возрасла.

Трёхмерные объекты математически представляются в виде множества точек - вершин, и в тоже время они, соединенные отрезками, образуют рёбра фигиры, а множество рёбер на одной плоскости образуют поверхности - грани фигуры. Данный массив точек трансформируется с учётом матрицы перспективы камеры, далее - накладываются текстуры на поверхности граней фигуры - изображения также растягивают и трансформируют с учётом перспективы наклона грани, затем - обрабатывают свет, падающий на модель, путём изменения яркости текстуры граней, в зависимости от источников света, и наконец - добавляют тень, которая отбрасывает данная трёхмерная модель.
\subsection{Работа с OpenGL}

"OpenGL" (Open Graphics Library) - это кроссплатформенная спецификация, независимая от языка программирования, которая определяет программный интерфейс для написания приложений для двумерной и трёхмерной компьютерной графики. По своей сути, OpenGL - это низкоуровневый API, который позволяет напрямую работать с коммандами и буфферами видеокарты, так что для его использования необходимо иметь хотя бы основные понимания работы трёхмерной графики и линейной алгебры. Для простого начала знакомства и использования спецификации OpenGL существует множество официальных готовых эффективных реализаций для Windows, Unix-платформ и MacOS.

Спецификация OpenGL была создана в эпоху активного распространения компьютерных 3D игр и приложений, когда взникла сложность с совместимостью устройств во время реализации компьютерного кода при использовании различных аппаратных устройств - процессоров, видеоадаптеров, устройств хранения информации и материнской платы. Это вызывало сильные затраты, трудности и замедляло разработку программного обеспечения. Поэтому, компания Silicon Graphics - лидирующая в то время в сфере производства оборудования для обработки трёхмерной графики разработала программный интерфейс, целью которого было систематизировать доступ и обработку трёхмерных моделей на аппаратном уровне. Плодами их трудов стала спецификация OpenGL, который стандартизировал обработку различных функций 3D систем, которые выполняли единые команды из списка доступных, согласно установленной программой спецификации, что позволило создавать программное обеспечение, которое одинаково корректно работает на всех видах графического оборудования.

Принцип работы  OpenGL заключается в получении геометрических векторных примитивов и построении растровой картинки в памяти видеокарты и выводе её на экран. Из-за своей низкоуровневости, данная спецификация требует диктовать программе точный порядок действий от программиста и использовать императивный подход в разработке приложений. Но несмотря на эти сложности, это даёт большой простор, гибкость и свободу в создании программ.

Последняя вышедшая версия OpenGL 4.6, в данный момент уже считается устаревшей, и на смену ей пришёл современный и оптимизированный для новых видеокарт API Vulcan, который стал прямым преемником OpenGL.