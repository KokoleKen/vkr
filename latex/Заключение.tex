\section*{ЗАКЛЮЧЕНИЕ}
\addcontentsline{toc}{section}{ЗАКЛЮЧЕНИЕ}

Задача трехмерной графики — презентовать объект, явление или пространство в стилизованной или реалистичной визуальной форме. Дизайнеры рисуют предметы почти с нуля, настраивают освещение, работают над композицией кадра и в целом выполняют большую работу, поэтому с каждым годом трёхмерная компьютерная графика становится реалистичнее.

Постоянное совершенствование компьютерного оборудования и программного обеспечения сделало 3D-технологии доступными. Сегодня 3D-модели повсеместно используют вместо обычных макетов в проектировании для проработки крупных или миниатюрных деталей, а «объемная» визуализация становится одним из инструментов маркетинговых мероприятий, интерактивных тренингов, презентаций.

Основные результаты работы:

\begin{enumerate}
\item Проведен анализ предметной области. Выявлена необходимость использовать библиотеку OpenGL.
\item Разработана концептуальная модель программы. Разработана модель данных системы. Определены требования к системе.
\item Осуществлено проектирование программной системы. Разработана основа графического движка. Разработан пользовательский интерфейс приложения.
\item Реализована и протестирована программная система. Проведено системное тестирование.
\end{enumerate}

Все требования, объявленные в техническом задании, были полностью реализованы, все задачи, поставленные в начале разработки проекта, были также решены.

Готовый рабочий проект представлен в виде десктопного приложения. Приложение записано на внешний носитель информации, который приложен к отчёту.  