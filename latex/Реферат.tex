\abstract{РЕФЕРАТ}

Объем работы равен \formbytotal{lastpage}{страниц}{е}{ам}{ам}. Работа содержит \formbytotal{figurecnt}{иллюстраци}{ю}{и}{й}, \formbytotal{tablecnt}{таблиц}{у}{ы}{}, \arabic{bibcount} библиографических источников и \formbytotal{числоПлакатов}{лист}{}{а}{ов} графического материала. Количество приложений – 2. Пример содержания файла геометрии объектов формата OBJ представлен в приложении А. Фрагменты исходного кода представлены в приложении Б.

Перечень ключевых слов: OpenGL, трёхмерный движок, виртуальная сцена, трёхмерная модель, текстура, вершины, грани, полигоны, триангулярность, диаграмма, объекты, файлы геометрии, парсер, трёхмерное пространство, освещение, координаты вершин, координаты текстур, направления нормалей, рендер, визуализация, проекция, камера, угол обзора, матрицы, матрица преобразований, матрица проекции камеры, аффинные преобразования.

Объектом разработки является программа, представляющая собой трёхмерный графический движок, работающий под управлением графической библиотеки и спецификации OpenGL, способный визуализировать загруженные в него данные трёхмерных объектов.

Целью выпускной квалификационной работы является создание программной основы в виде графического трёхмерного движка с открытым кодом для дальнейших разработок в сфере трёхмерного моделирования и гейм-дизайна.

В процессе создания программного обеспечения были выделены основные сущности путем создания информационных блоков, использованы классы и методы модулей, обеспечивающие работу с сущностями предметной области, а также корректную работу программной системы, разработана программа для визуализации трёхмерных данных.

При разработке программной системы использовалась графическая библиотека и спецификация "<OpenGL">.

\selectlanguage{english}
\abstract{ABSTRACT}
  
The volume of work is \formbytotal{lastpage}{page}{}{s}{s}. The work contains \formbytotal{figurecnt}{illustration}{}{s}{s}, \formbytotal{tablecnt}{table}{}{s}{s}, \arabic{bibcount} bibliographic sources and \formbytotal{числоПлакатов}{sheet}{}{s}{s} of graphic material. The number of applications is 2. The graphic material is presented in annex A. The layout of the site, including the connection of components, is presented in annex Г.

List of keywords: OpenGL, three-dimensional engine, virtual scene, three-dimensional model, texture, vertices, faces, polygons, triangularity, diagram, objects, geometry files, parser, three-dimensional space, lighting, vertex coordinates, texture coordinates, normal directions, render, visualization, projection, camera, viewing angle, matrices, transformation matrix, camera projection matrix, affine transformations.

The object of development is a program that is a three-dimensional graphics engine running under the control of a graphics library and the OpenGL specification, capable of visualizing the data of three-dimensional objects loaded into it.

The goal of the final qualifying work is to create a software basis in the form of an open-source 3D graphic engine for further developments in the field of 3D modeling and game design.

In the process of creating the software, the main entities were identified by creating information blocks, classes and methods of modules were used to ensure work with entities of the subject area, as well as the correct operation of the software system, and a program for visualizing three-dimensional data was developed.

When developing the software system, the graphics library and specification "<OpenGL"> were used.
\selectlanguage{russian}
