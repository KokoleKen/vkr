\section*{ВВЕДЕНИЕ}
\addcontentsline{toc}{section}{ВВЕДЕНИЕ}

Трёхмерная компьютерная графика появилась в 1960-х годах. Первые векторные изображения состояли из множества точек и кривых, заданных математической формулой. Айван Сазерленд и Дэвид Эванс основали первую в мире кафедру компьютерной графики в университете Юты, США. Именно в те времена появилась программа Sketchpad (от англ. «альбом для рисования») — предок всех современных 3D-редакторов.

На первых компьютерах можно было работать только с векторными изображениями. Затем появилась растровая графика, которая позволила изображать объекты в виде сетки пикселей. Кроме сплошных цветов и фигур, начали использовать текстуры и тени, расширились возможности рендеринга — превращения кода в финальные изображения.

Трёхмерные объекты математически представляются в виде множества точек - вершин, и в тоже время они, соединенные отрезками, образуют рёбра фигиры, а множество рёбер на одной плоскости образуют поверхности - грани фигуры. Данный массив точек трансформируется с учётом матрицы перспективы камеры, далее - накладываются текстуры на поверхности граней фигуры - изображения также растягивают и трансформируют с учётом перспективы наклона грани, затем - обрабатывают свет, падающий на модель, путём изменения яркости текстуры граней, в зависимости от источников света, и наконец - добавляют тень, которая отбрасывает данная трёхмерная модель.

3D-движок, рендерер или визуализатор является программной системой графического движка (название произошло от английского graphics engine) — промежуточное программное обеспечение, программный движок, основной задачей которого является визуализация двухмерной и трёхмерной компьютерной графики. Может существовать как отдельный продукт или в составе игрового движка. Может использоваться для визуализации отдельных изображений или компьютерного видео. Графические движки, использующееся в программах по работе с компьютерной графикой обычно называются «рендерерами», «отрисовщиками» или «визуализаторами». Само название «графический движок» используется, как правило, в компьютерных играх.

Основное и важнейшее отличие «игровых» графических движков от неигровых состоит в том, что первые должны обязательно работать в режиме реального времени, тогда как вторые могут тратить по несколько десятков часов на вывод одного изображения. Вторым существенным отличием является то, что начиная приблизительно с 1995-1997 года, графические движки производят визуализацию с помощью графических процессоров, которые установлены на отдельных платах — видеокартах. Программные графические движки используют только центральные процессоры.

Как правило, графические движки не распространяются отдельно от игровых. Единственного графического движка без дополнительных компонентов и инструментария недостаточно для создания игры, поэтому разработчики движков продают лишь игровые движки с полным набором инструментов и компонентов. Однако это правило не относится к свободному программному обеспечению. Энтузиасты создают свободные графические движки и свободно их распространяют. Впоследствии разработчики игр могут объединить свободный графический движок с физическим, звуковым и другими компонентами и создать на основе их полноценный игровой движок.

\emph{Цель настоящей работы} – разработка графического движка - программной системы реализующей визуализацию трёхмерных данных, на основе графической библиотеки OpenGL. Для достижения поставленной цели необходимо решить \emph{следующие задачи:}
\begin{itemize}
\item провести анализ предметной области;
\item разработать концептуальную модель программы;
\item спроектировать программную систему;
\item реализовать программную среду используя графическую библиотеку OpenGL.
\end{itemize}

\emph{Структура и объем работы.} Отчет состоит из введения, 4 разделов основной части, заключения, списка использованных источников, 2 приложений. Текст выпускной квалификационной работы равен \formbytotal{lastpage}{страниц}{е}{ам}{ам}.

\emph{Во введении} сформулирована цель работы, поставлены задачи разработки, описана структура работы, приведено краткое содержание каждого из разделов.

\emph{В первом разделе} на стадии описания технической характеристики предметной области приводится сбор информации о деятельности компании, для которой осуществляется разработка сайта.

\emph{Во втором разделе} на стадии технического задания приводятся требования к разрабатываемому приложению.

\emph{В третьем разделе} на стадии технического проектирования представлены проектные решения для программы.

\emph{В четвертом разделе} приводится список классов и их методов, использованных при разработке программной системы, производится тестирование разработанного сайта.

В заключении излагаются основные результаты работы, полученные в ходе разработки.

В приложении А представлен пример содержания файла описании геометрии трёхмерных объектов в формате OBJ.


В приложении Г представлены фрагменты исходного кода. 
